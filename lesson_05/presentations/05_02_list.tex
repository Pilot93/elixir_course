\documentclass[10pt]{beamer}

\usepackage{fontspec}
\setmainfont{Ubuntu}[]
\setsansfont{Ubuntu}[]
\setmonofont{Ubuntu Mono}[]

\usepackage{graphicx}
\graphicspath{ {../img/} }

\usepackage{listings}

%% \usepackage[absolute,overlay]{textpos}[showboxes]

\beamertemplatenavigationsymbolsempty

\title{Устройство списков}

\begin{document}

\begin{frame}[fragile]
  \frametitle{Создание списка}
  \begin{lstlisting}
    a + b = 42
  \end{lstlisting}
\end{frame}

\begin{frame}
  \frametitle{Создание списка}
  \centering
  \includegraphics[scale=0.6]{list_1}
\end{frame}

\begin{frame}
  \frametitle{Добавление нового элемента}
  \centering
  \includegraphics[scale=0.6]{list_2}
\end{frame}

\begin{frame}
  \frametitle{Список как последовательность операторов cons}
  \centering
  \includegraphics[scale=0.6]{list_3}
\end{frame}

\begin{frame}
  \frametitle{Создание двунаправленного списка}
  \centering
  \includegraphics[scale=0.6]{list_4}
\end{frame}

\begin{frame}
  \frametitle{Добавление нового элемента}
  \centering
  \includegraphics[scale=0.6]{list_5}
\end{frame}

\begin{frame}
  \frametitle{Полное копирование списка}
  \centering
  \includegraphics[scale=0.6]{list_6}
\end{frame}

\end{document}
